\documentclass[12pt]{article}
\usepackage{amssymb}
\usepackage{scrextend}
\usepackage[utf8]{inputenc}
\usepackage[polish]{babel}
\usepackage[T1]{fontenc}%polskie znaki
\usepackage[utf8]{inputenc}%polskie znaki
\usepackage{geometry}
\usepackage{float}
\usepackage{enumitem}
\usepackage{hyperref}
\usepackage{graphicx}
\usepackage{tabulary}
\usepackage{etoc}
\usepackage[normalem]{ulem} 
\renewcommand{\baselinestretch}{1.5}
\graphicspath{ {img/} }
\newgeometry{lmargin=2.0cm, rmargin=2.0cm, tmargin=2.0cm, bmargin=2.0cm}
\usepackage{tikz}
\usepackage[bf, tablename=Tabela]{caption}
\usepackage{pgfplots}
\usepackage{csvsimple}
\usepackage{pgfplotstable}
\usepackage{siunitx}
\pgfplotsset{compat=1.16}
\usepackage{amsmath}
\usepackage{braket}
\usepackage{listings}
\usepackage{breqn}
\usepackage[normalem]{ulem} 
\definecolor{codegreen}{rgb}{0,0.6,0}
\definecolor{codegray}{rgb}{0.5,0.5,0.5}
\definecolor{codepurple}{rgb}{0.58,0,0.82}
\definecolor{backcolour}{rgb}{0.99,0.99,0.98}
 
\lstdefinestyle{mystyle}{
    backgroundcolor=\color{backcolour},   
    commentstyle=\color{codegreen},
    keywordstyle=\color{magenta},
    numberstyle=\tiny\color{codegray},
    stringstyle=\color{codepurple},
    basicstyle=\ttfamily\footnotesize,
    breakatwhitespace=false,         
    breaklines=true,                 
    captionpos=b,                    
    keepspaces=true,                 
    numbers=left,                    
    numbersep=5pt,                  
    showspaces=false,                
    showstringspaces=false,
    showtabs=false,                  
    tabsize=2
}
\lstset{style=mystyle}

\title{ 
    \vspace*{50mm}
    \textsc{
        \textbf{Projektowanie efektywnych algorytmów}\\
    \vspace*{10mm}
    \large Projekt\\
        }
        \normalsize
         Algorytm genetyczny
         \vspace*{5mm}
         } 
\author{
Damian Koper,  241292\\
}

\date{\today}

\begin{document}

\maketitle

\newpage
\setcounter{tocdepth}{2}
\localtableofcontents
\listoffigures
\listoftables
\vfill
Kod i wersje wykonywalne programów: \url{https://github.com/damiankoper/peaProject1}
\newpage

\section{Wstęp}
Celem projektu było wykonanie programu, który rozwiązywał będzie asymetryczny
problem komiwojażera z wykorzystaniem algorytmu genetycznego.
\section{Problem komiwojażera}
Asymetryczny problem komiwojażera (ATSP - Asynchronous Travelling Salesman Problem) jest problemem optymalizacyjnym należącym do klasy NP-trudnych.
Polega on na znalezieniu najkrótszego cyklu Hamiltona w skierowanym grafie ważonym. Instancja problemu reprezentowana jest przez macierz sąsiedztwa, która na
przekątnej ma wartości $-1$.

\subsection{Algorytm genetyczny}
Algorytm genetyczny bazuje na zjawisku ewolucji w przyrodzie, gdzie każde kolejne pokolenie gatunku jest lepiej przystosowane do warunków panujących w środowisku.
Wynika to z faktu, że tylko najlepiej przystosowane osobniki mają szansę na przeżycie. W procesie rozmnażania osobniki przekazują swoje przystosowanie potomstwu
tworząc osobnika lepiej przystosowanego, co nie zawsze jednak musi mieć miejsce. U potomstwa mogą również wystąpić losowe zmiany - mutacje, które zmieniają ich przystosowanie na lepsze lub gorsze.


Podstawowe pojęcia związane z algorytmem genetycznym:
\begin{itemize}[noitemsep]
    \item \textbf{populacja}: zbiór osobników określonej liczebności.
    \item \textbf{osobnik}: zakodowany w postaci chromosomów zbiór parametrów zadania - rozwiązanie, punkty przeszukiwanej przestrzeni. Dla TSP jest to ścieżka i jej długość.
    \item \textbf{chromosom(y)}: ciągi kodowe składające się z genów. W TSP analizie podlega tylko jeden chromosom.
    \item \textbf{gen}: cecha, pojedynczy element genotypu.
    \item \textbf{genotyp}: zespół chromosomów osobnika (struktura osobnika). Osobniki mogą być genotypami lub chromosomami, jeśli gentotyp składa się tylko z jednego chromosomu.
    \item \textbf{fenotyp}: zestaw wartości odpowiadających danemu genotypowi.
    \item \textbf{allel}: wartość danego genu (wartość cechy).
    \item \textbf{locus}: pozycja wskazująca miejsce położenia danego genu w ciągu (chromosomie)
\end{itemize}

\subsubsection{Selekcja}
Operacja selekcji określa, które z osobników, po operacji krzyżowania i mutacji, zostaną wybrane jako te najlepiej przystosowane.
Na ich podstawie tworzone będą kolejne pokolenia osobników. W zaimplementowanym algorytmie do kolejnej iteracji wybierane są
najlepsze osobniki w ilości określonej przez rozmiar populacji.

\subsubsection{Krzyżowanie}
Operacja krzyżowania polega na wymianie materiału genetycznego pomiędzy losowo wybranymi w procesie selekcji rodzicami.
Osobnik powstały w wyniku krzyżowania powinien być lepiej przystosowany od swoich rodziców. Krzyżowanie zachodzi z
ustalonym prawdopodobieństwem $p_c$. W algorytmie zaimplementowano następujące operatory krzyżowania:
\begin{itemize}
    \item \textbf{PMX} - Partially Mapped Crossover (Goldberg, 1985)\\
    Polega na wytworzeniu i zastosowania mapowania pomiędzy przeniesionymi elementami pierwszego rodzica z tymi elementami rodzica drugiego, które występują w przeniesionym fragmencie.
    \\
        Rodzic 1: \textcolor{blue}{8 4 7 \textbf{3 6 2 5 1} 9 0}
        \\
        Rodzic 2: \textcolor{red}{0 1 2 \textbf{3 4 5 6 7} 8 9}\\
        Dziecko : \textcolor{red}{0 \textbf{7 4}} \textcolor{blue}{\textbf{3 6 2 5 1}} \textcolor{red}{8 9}
    \item \textbf{OX} - OX - Ordered Crossover (Davis, 1985)\\
    Polega na przeniesieniu brakujących elementów w kolejności w jakiej występują w drugim rodzicu.
    \\
    Rodzic 1: \textcolor{blue}{8 4 7 \textbf{3 6 2 5 1} 9 0}
    \\
    Rodzic 2: \textcolor{red}{0 \sout{1 2} \textbf{3 4 5 6 7} 8 9}\\
    Dziecko : \textcolor{red}{0 \textbf{4 7}} \textcolor{blue}{\textbf{3 6 2 5 1}} \textcolor{red}{8 9}
\end{itemize}

\clearpage

\subsubsection{Mutacje}
Mutacje są elementem pozwalającym na zwiększenie szansy na wyjście z optimów lokalnych podczas przeszukiwania.
Polegają one na niewielkiej zmianie w genotypie osobnika. Zachodzi ona z ustalonym prawdopodobieństwie $p_m$, które nie powinno
być większe niż $0.1$. W zaimplementowanym algorytmie mutacja jest przeprowadzona z użyciem operatora sąsiedztwa typu \textit{Insert}.

\begin{equation}
    \label{eq:3}
    \pi=<\pi(1),...,\pi(i-1),\mathbin{\color{red}\pi(i)},\mathbin{\color{blue}\pi(j)},\pi(i+1),...,\pi(j-1),\pi(j+1),...,\pi(n)>
\end{equation}

\section{Pomiary}
Algorytmy i system pomiarowy zostały zaimplementowane wykorzystując podejście obiektowe, elementy biblioteki \textit{STL} i wersję \textit{C++17}.
Pomiarom podlegał wynik w funkcji czasu określonym liczbą iteracji. Zaimplementowane algorytmy nie posiadają funkcji pomiaru czasu. Bazują one na liczbie iteracji wykonanych do chwili pomiaru.

Pomiary przeprowadzone były dla plików:
\begin{itemize}[noitemsep]
    \item ftv47.atsp
    \item ftv170.atsp
    \item rbg403.atsp
\end{itemize}
Dla każdego z nich uruchomiono algorytm genetyczny. Wynik generowany metodą zachłanną był zawsze umieszczany w generacji początkowej. Pozostałe osobniki były losowane.
Użyto operatorów krzyżowania \textit{PMX} i \textit{OX} z prawdopodobieństwem $p_c=0.8$ i operatora sąsiedztwa dla mutacji typu \textit{Insert} z prawdopodobieństwem $p_m=0.01$.
\clearpage
\subsection{Wyniki}
\subsubsection{Plik ftv47 - niektóre wykonania}
\begin{multicols}{3}

\begin{table}[H]
    \centering
    \begin{tabular}{|r|r|}
        \hline
        \textbf{Iteracja} & \textbf{Błąd ${[\%]}$}
        \csvreader[head to column names, filter expr={
            test{\ifnumless{\thecsvinputline}{25}}
        }]
        {./../../GA_48_100_pmx_0.8_0.01.csv}{}
        {\\\hline\csvcoli & \csvcoliii}\\
        \hline
    \end{tabular}
    \caption{Błąd w kolejnych iteracjach dla pliku ftv47. $P=100$, PMX, $p_c=0.8$, $p_m=0.01$.}
\end{table}
\begin{table}[H]
    \centering
    \begin{tabular}{|r|l|}
        \hline
        \textbf{Iteracja} & \textbf{Błąd ${[\%]}$}
        \csvreader[head to column names, filter expr={
            test{\ifnumless{\thecsvinputline}{25}}
        }]
        {./../../GA_48_100_ox_0.8_0.01.csv}{}
        {\\\hline\csvcoli & \csvcoliii}\\
        \hline
    \end{tabular}
    \caption{Błąd w kolejnych iteracjach dla pliku ftv47. $P=100$, OX, $p_c=0.8$, $p_m=0.01$.}
\end{table}
\columnbreak
\begin{table}[H]
    \centering
    \begin{tabular}{|r|l|}
        \hline
        \textbf{Iteracja} & \textbf{Błąd ${[\%]}$}
        \csvreader[head to column names, filter expr={
            test{\ifnumless{\thecsvinputline}{25}}
        }]
        {./../../GA_48_350_pmx_0.8_0.01.csv}{}
        {\\\hline\csvcoli & \csvcoliii}\\
        \hline
    \end{tabular}
    \caption{Błąd w kolejnych iteracjach dla pliku ftv47. $P=350$, PMX, $p_c=0.8$, $p_m=0.01$.}
\end{table}
\columnbreak
\begin{table}[H]
    \centering
    \begin{tabular}{|r|l|}
        \hline
        \textbf{Iteracja} & \textbf{Błąd ${[\%]}$}
        \csvreader[head to column names, filter expr={
            test{\ifnumless{\thecsvinputline}{25}}
        }]
        {./../../GA_48_350_ox_0.8_0.01.csv}{}
        {\\\hline\csvcoli & \csvcoliii}\\
        \hline
    \end{tabular}
    \caption{Błąd w kolejnych iteracjach dla pliku ftv47. $P=350$, OX, $p_c=0.8$, $p_m=0.01$.}
\end{table}
\end{multicols}

\subsubsection{ftv47}
\begin{figure}[H]
    \centering
    \begin{tikzpicture}
        \begin{axis}[
        width=0.9\linewidth,
        height=7cm,
        grid=major,
        xlabel=Czas{[iteracje]},
        ylabel=Błąd{[\%]},
        legend style={anchor=south east, at={(1,0)}},
        xmax=7000,
        xmin=-100
        ]
        \addplot table [x index=0, y index=2, col sep=comma] {./../../GA_48_100_pmx_0.8_0.01.csv};
        \addlegendentry{100 PMX}
        \addplot table [x index=0, y index=2, col sep=comma] {./../../GA_48_100_ox_0.8_0.01.csv};
        \addlegendentry{100 OX}
        \addplot table [x index=0, y index=2, col sep=comma] {./../../GA_48_350_pmx_0.8_0.01.csv};
        \addlegendentry{350 PMX}
        \addplot table [x index=0, y index=2, col sep=comma] {./../../GA_48_350_ox_0.8_0.01.csv};
        \addlegendentry{350 OX}
        \addplot table [x index=0, y index=2, col sep=comma] {./../../GA_48_600_pmx_0.8_0.01.csv};
        \addlegendentry{600 PMX}
        \addplot table [x index=0, y index=2, col sep=comma] {./../../GA_48_600_ox_0.8_0.01.csv};
        \addlegendentry{600 OX}
        \end{axis}
    \end{tikzpicture}
    \caption{Wykres dla pliku ftv47.}
    \label{plot:1}
\end{figure}

Najlepsza ścieżka została znaleziona w iteracji \textbf{1464} z błędem \textbf{9.85\%}.

\subsubsection{ftv170}
\begin{figure}[H]
    \centering
    \begin{tikzpicture}
        \begin{axis}[
        width=0.9\linewidth,
        height=7cm,
        grid=major,
        xlabel=Czas{[iteracje]},
        ylabel=Błąd{[\%]},
        legend style={anchor=south west, at={(0,0)}},
        xmin=-100
        ]
        \addplot table [x index=0, y index=2, col sep=comma] {./../../GA_171_100_pmx_0.8_0.01.csv};
        \addlegendentry{100 PMX}
        \addplot table [x index=0, y index=2, col sep=comma] {./../../GA_171_100_ox_0.8_0.01.csv};
        \addlegendentry{100 OX}
        \addplot table [x index=0, y index=2, col sep=comma] {./../../GA_171_350_pmx_0.8_0.01.csv};
        \addlegendentry{350 PMX}
        \addplot table [x index=0, y index=2, col sep=comma] {./../../GA_171_350_ox_0.8_0.01.csv};
        \addlegendentry{350 OX}
        \addplot table [x index=0, y index=2, col sep=comma] {./../../GA_171_600_pmx_0.8_0.01.csv};
        \addlegendentry{600 PMX}
        \addplot table [x index=0, y index=2, col sep=comma] {./../../GA_171_600_ox_0.8_0.01.csv};
        \addlegendentry{600 OX}
        \end{axis}
    \end{tikzpicture}
    \caption{Wykres dla pliku ftv170.}
    \label{plot:2}
\end{figure}

Najlepsza ścieżka została znaleziona w iteracji \textbf{65429} z błędem \textbf{34.7\%}.
Instancja problemu TSP zawarta w tym pliku jest specyficzna i źle obsługiwana przez zaimplementowane algorytmy w tym i poprzednich projektach.

\subsubsection{rbg403}
\begin{figure}[H]
    \centering
    \begin{tikzpicture}
        \begin{axis}[
        width=0.9\linewidth,
        height=7cm,
        grid=major,
        xlabel=Czas{[iteracje]},
        ylabel=Błąd{[\%]},
        legend style={anchor=north west, at={(0,0)}},
        xmax=51000,
        xmin=-100
        ]
        \addplot table [x index=0, y index=2, col sep=comma] {./../../GA_403_100_pmx_0.8_0.01.csv};
        \addlegendentry{100 PMX}
        \addplot table [x index=0, y index=2, col sep=comma] {./../../GA_403_100_ox_0.8_0.01.csv};
        \addlegendentry{100 OX}
        \addplot table [x index=0, y index=2, col sep=comma] {./../../GA_403_350_pmx_0.8_0.01.csv};
        \addlegendentry{350 PMX}
        \addplot table [x index=0, y index=2, col sep=comma] {./../../GA_403_350_ox_0.8_0.01.csv};
        \addlegendentry{350 OX}
        \addplot table [x index=0, y index=2, col sep=comma] {./../../GA_403_600_pmx_0.8_0.01.csv};
        \addlegendentry{600 PMX}
        \addplot table [x index=0, y index=2, col sep=comma] {./../../GA_403_600_ox_0.8_0.01.csv};
        \addlegendentry{600 OX}
        \end{axis}
    \end{tikzpicture}
    \caption{Wykres dla pliku rbg403.}
    \label{plot:3}
\end{figure}

Najlepsza ścieżka została znaleziona w iteracji \textbf{45510} z błędem \textbf{6.69\%}.

\clearpage
\subsection{Pomiary dla zmiennego prawdopodobieństwa krzyżowania}
\subsubsection{ftv47}
\begin{figure}[H]
    \centering
    \begin{tikzpicture}
        \begin{axis}[
        width=0.9\linewidth,
        height=7cm,
        grid=major,
        xlabel=Czas{[iteracje]},
        ylabel=Błąd{[\%]},
        legend style={anchor=south east, at={(1,0)}},
        xmin=-100
        ]
        \addplot table [x index=0, y index=2, col sep=comma] {./../../GA_48_350_pmx_0.5_0.01.csv};
        \addlegendentry{350 PMX, $p_c=0.5$}
        \addplot table [x index=0, y index=2, col sep=comma] {./../../GA_48_350_pmx_0.7_0.01.csv};
        \addlegendentry{350 PMX, $p_c=0.7$}
        \addplot table [x index=0, y index=2, col sep=comma] {./../../GA_48_350_pmx_0.8_0.01.csv};
        \addlegendentry{350 PMX, $p_c=0.8$}
        \addplot table [x index=0, y index=2, col sep=comma] {./../../GA_48_350_pmx_0.9_0.01.csv};
        \addlegendentry{350 PMX, $p_c=0.9$}        
        \end{axis}
    \end{tikzpicture}
    \caption{Wykres dla pliku ftv47.}
    \label{plot:1}
\end{figure}

\subsubsection{ftv170}
\begin{figure}[H]
    \centering
    \begin{tikzpicture}
        \begin{axis}[
        width=0.9\linewidth,
        height=7cm,
        grid=major,
        xlabel=Czas{[iteracje]},
        ylabel=Błąd{[\%]},
        legend style={anchor=south west, at={(0,0)}},
        xmin=-100
        ]
        \addplot table [x index=0, y index=2, col sep=comma] {./../../GA_171_100_ox_0.5_0.01.csv};
        \addlegendentry{100 OX, $p_c=0.5$}
        \addplot table [x index=0, y index=2, col sep=comma] {./../../GA_171_100_ox_0.7_0.01.csv};
        \addlegendentry{100 OX, $p_c=0.7$}
        \addplot table [x index=0, y index=2, col sep=comma] {./../../GA_171_100_ox_0.8_0.01.csv};
        \addlegendentry{100 OX, $p_c=0.8$}
        \addplot table [x index=0, y index=2, col sep=comma] {./../../GA_171_100_ox_0.9_0.01.csv};
        \addlegendentry{100 OX, $p_c=0.9$}        
        \end{axis}
    \end{tikzpicture}
    \caption{Wykres dla pliku ftv170.}
    \label{plot:1}
\end{figure}

\subsubsection{rbg403}
\begin{figure}[H]
    \centering
    \begin{tikzpicture}
        \begin{axis}[
        width=0.9\linewidth,
        height=7cm,
        grid=major,
        xlabel=Czas{[iteracje]},
        ylabel=Błąd{[\%]},
        legend style={anchor=north east, at={(1,1)}},
        xmin=-100
        ]
        \addplot table [x index=0, y index=2, col sep=comma] {./../../GA_403_350_pmx_0.5_0.01.csv};
        \addlegendentry{350 PMX, $p_c=0.5$}
        \addplot table [x index=0, y index=2, col sep=comma] {./../../GA_403_350_pmx_0.7_0.01.csv};
        \addlegendentry{350 PMX, $p_c=0.7$}
        \addplot table [x index=0, y index=2, col sep=comma] {./../../GA_403_350_pmx_0.8_0.01.csv};
        \addlegendentry{350 PMX, $p_c=0.8$}
        \addplot table [x index=0, y index=2, col sep=comma] {./../../GA_403_350_pmx_0.9_0.01.csv};
        \addlegendentry{350 PMX, $p_c=0.9$}        
        \end{axis}
    \end{tikzpicture}
    \caption{Wykres dla pliku rbg403.}
    \label{plot:1}
\end{figure}

\section{Podsumowanie}

\begin{table}[H]
    \centering
    \begin{tabular}{|l|c|c|}
        \hline
        Plik & Tabu Search & Genetic Algorithm\\
        \hline
        ftv48 & 0.17\% & 9.63\% \\
        ftv170 & 11.36\% & 34.7\% \\ 
        rbg403 & 0.57\% & 6.69\%\\
        \hline
    \end{tabular}
    \caption{Porównanie Tabu Search z algorytmem genetycznym}
\end{table}
Z dwóch analizowanych metod zdecydowanie lepiej radzi sobie metoda TabuSearch.
Prawdopodobieństwo krzyżowania ma duży wpływ na szybkość uzyskania lepszych wyników. Operator \textit{PMX} prawie we wszystkich przypadkach
radzi sobie lepiej niż operator \textit{OX}. Rozmiar populacji powinien zależeć od rozmiaru problemu. Zbyt mały nie pozwala wygenerować odpowiednio
zróżnicowanej populacji, a zbyt duży generuje niepotrzebną konieczność obliczeń. Prawdopodobieństwo krzyżowania również powinno być odpowiednio dobrane, ponieważ zbyt niskie nie pozwoli populacji uzyskać dużej ilości potomnych osobników.
Zbyt wysokie nie pozwoli przetrwać rodzicom, którzy mogą być podstawą lepszego rozwiązania w późniejszym krzyżowaniu.

W algorytmie genetycznym widać większy wpływ czynnika losowego.


\end{document}