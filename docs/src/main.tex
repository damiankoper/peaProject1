\documentclass[12pt]{article}
\usepackage{amssymb}
\usepackage{scrextend}
\usepackage[utf8]{inputenc}
\usepackage[polish]{babel}
\usepackage[T1]{fontenc}%polskie znaki
\usepackage[utf8]{inputenc}%polskie znaki
\usepackage{geometry}
\usepackage{float}
\usepackage{enumitem}
\usepackage{hyperref}
\usepackage{graphicx}
\usepackage{tabulary}
\usepackage{etoc}
\usepackage[normalem]{ulem} 
\renewcommand{\baselinestretch}{1.5}
\graphicspath{ {img/} }
\newgeometry{lmargin=2.0cm, rmargin=2.0cm, tmargin=2.0cm, bmargin=2.0cm}
\usepackage{tikz}
\usepackage[bf, tablename=Tabela]{caption}
\usepackage{pgfplots}
\usepackage{csvsimple}
\usepackage{pgfplotstable}
\usepackage{siunitx}
\pgfplotsset{compat=1.16}
\usepackage{amsmath}
\usepackage{braket}
\usepackage{listings}
\usepackage{breqn}

\definecolor{codegreen}{rgb}{0,0.6,0}
\definecolor{codegray}{rgb}{0.5,0.5,0.5}
\definecolor{codepurple}{rgb}{0.58,0,0.82}
\definecolor{backcolour}{rgb}{0.99,0.99,0.98}
 
\lstdefinestyle{mystyle}{
    backgroundcolor=\color{backcolour},   
    commentstyle=\color{codegreen},
    keywordstyle=\color{magenta},
    numberstyle=\tiny\color{codegray},
    stringstyle=\color{codepurple},
    basicstyle=\ttfamily\footnotesize,
    breakatwhitespace=false,         
    breaklines=true,                 
    captionpos=b,                    
    keepspaces=true,                 
    numbers=left,                    
    numbersep=5pt,                  
    showspaces=false,                
    showstringspaces=false,
    showtabs=false,                  
    tabsize=2
}
\lstset{style=mystyle}

\title{ 
    \vspace*{50mm}
    \textsc{
        \textbf{Projektowanie efektywnych algorytmów}\\
    \vspace*{10mm}
    \large Projekt\\
        }
        \normalsize
         Symulowane wyżarzanie i TabuSearch
         \vspace*{5mm}
         } 
\author{
Damian Koper,  241292\\
}

\date{\today}

\begin{document}

\maketitle

\newpage
\setcounter{tocdepth}{2}
\localtableofcontents
\listoffigures
\listoftables
\vfill
Kod i wersje wykonywalne programów: \url{https://github.com/damiankoper/peaProject1}
\newpage

\section{Wstęp}
Celem projektu było wykonanie programu, który rozwiązywał będzie asymetryczny
problem komiwojażera z wykorzystaniem metod symulowanego wyżarzania i \textit{TabuSearch}.
\section{Problem komiwojażera i przeszukiwanie lokalne}
Asymetryczny problem komiwojażera (ATSP - Asynchronous Travelling Salesman Problem) jest problemem optymalizacyjnym należącym do klasy NP-trudnych.
Polega on na znalezieniu najkrótszego cyklu Hamiltona w skierowanym grafie ważonym. Instancja problemu reprezentowana jest przez macierz sąsiedztwa, która na
przekątnej ma wartości $-1$.

Przeszukiwanie lokalne polega na analizowaniu tylko najbliższego sąsiedztwa obecnego rozwiązania. Rozwiązanie to nie zawsze musi być
najlepszym dotychczas znalezionym, ponieważ inne, z pozoru gorsze, mogą okazać się drogą do rozwiązania globalnie najlepszego.

\subsection{Definicje sąsiedztwa}
Sąsiedztwo definiowane może być na wiele dowolnych sposobów. Ważne jest, żeby proces generowania najbliższego sąsiada miał wielomianową złożoność obliczeniową.
W problemie komiwojażera wynikiem jest ścieżka określana poprzez permutację wierzchołków. Metody generowania sąsiadów będą wprowadzać modyfikacje do bazowej permutacji ze wzoru \ref{eq:1}.

\begin{equation}
    \label{eq:1}
    \pi=<\pi(1),...,\pi(i-1),\mathbin{\color{red}\pi(i)},\pi(i+1),...,\pi(j-1),\mathbin{\color{blue}\pi(j)},\pi(j+1),...,\pi(n)>
\end{equation}

\subsubsection{Swap}
Sąsiedztwo typu \textit{Swap} generowane jest poprzez zamianę dwóch elementów miejscami.
\begin{equation}
    \label{eq:2}
    \pi=<\pi(1),...,\pi(i-1),\mathbin{\color{blue}\pi(j)},\pi(i+1),...,\pi(j-1),\mathbin{\color{red}\pi(i)},\pi(j+1),...,\pi(n)>
\end{equation}
\subsubsection{Insert}
Sąsiedztwo typu \textit{Insert} generowane jest poprzez wybranie jednego elementu i wstawienie go za drugi wybrany element.
\begin{equation}
    \label{eq:3}
    \pi=<\pi(1),...,\pi(i-1),\mathbin{\color{red}\pi(i)},\mathbin{\color{blue}\pi(j)},\pi(i+1),...,\pi(j-1),\pi(j+1),...,\pi(n)>
\end{equation}
\subsubsection{Reverse}
Sąsiedztwo typu \textit{Reverse} generowane jest poprzez odwrócenie kolejności elementów między wybranymi dwoma elementami.
\begin{equation}
    \label{eq:4}
    \pi=<\pi(1),...,\pi(i-1),\mathbin{\color{blue}\pi(j)},\pi(j-1),...,\pi(i+1),\mathbin{\color{red}\pi(i)},\pi(j+1),...,\pi(n)>
\end{equation}
\subsection{Symulowane wyżarzanie}

Symulowane wyżarzanie jest metodą, która inspiruje się metalurgią, gdzie powolne wychładzanie stopu pozwala uporządkować jego strukturę, co przekłada się na jego lepsze właściwości.
Dla problemów optymalizacyjnych polega ona na warunkowym akceptowaniu gorszych rozwiązań, które w przyszłości mogą doprowadzić do lepszego wyniku. Warunek ten określa temperatura, ustalone prawdopodobieństwo i różnica pomiędzy wyszukanym rozwiązaniem, a poprzednim.
Rozwiązanie to akceptowane jest jeśli jest lepsze niż ostatnie lub obliczona wartość $p$ ze wzoru \ref{eq:5} jest większa niż ustalone prawdopodobieństwo akceptacji. Kolejne rozwiązanie jest generowane losowo używając jednej z definicji sąsiedztwa.
\begin{equation}
    \label{eq:5}
    p = \exp^{\left(\dfrac{f(x)-f(y)}{T}\right)}\\
\end{equation}
gdzie:\\
$f$ - funkcja długości drogi rozwiązania\\
$x$ - rozwiązanie poprzednie\\
$y$ - rozwiązanie wygenerowane, analizowane\\
$T$ - temperatura

Temperatura wpływa na prawdopodobieństwo zaakceptowania nowego gorszego rozwiązania. Gdy jest ona większa prawdopodobieństwo to jest większe. Po każdej iteracji temperatura jest zmniejszana zgodnie ze wzorem $T(t+1)=T(t)*a$, gdzie $a$ jest liczbą z przedziału (0;1). Warunkiem zakończenia algorytmu jest osiągnięcie temperatury końcowej $T_k$.

\subsection{TabuSearch}
TabuSearch (\textit{Przeszukiwanie z zakazami}) jest metodą przeszukiwania lokalnego, która opiera się na możliwości zakazania określonego ruchu na pewien czas.
Zakaz ten pozwala uniknąć algorytmowi \textit{cofania się} i powracania do przeanalizowanych już obszarów.
Do listy tabu dodaje się ruch, który spełnia kryteria aspiracji. W przypadku implementowanego algorytmu kryterium tym było wygenerowanie wyniku lepszego niż ostatni. Parametr kadencji określa ile iteracji dodany ruch może znajdować się na liście tabu.
Może zdarzyć się sytuacja, w której algorytm utknie w pewnym minimum lokalnym, dalekim od optymalnego wyniku. Przypadek ten określa kryterium dywersyfikacji. Kiedy jest ono spełnione, algorytm może zacząć poszukiwania w innym miejscu przestrzeni rozwiązań.


\section{Pomiary}
Algorytmy i system pomiarowy zostały zaimplementowane wykorzystując podejście obiektowe, elementy biblioteki \textit{STL} i wersję \textit{C++17}.
Pomiarom podlegał wynik w funkcji czasu. Zaimplementowane algorytmy nie posiadają funkcji pomiaru czasu. Bazują one na liczbie iteracji wykonanych do chwili pomiaru.

Pomiary przeprowadzone były dla plików:
\begin{itemize}[noitemsep]
    \item ftv47.atsp
    \item ftv170.atsp
    \item rbg403.atsp
\end{itemize}
Dla każdego z nich uruchomiono algorytm symulowanego wyżarzania i TabuSearch. wynik początkowy generowany był metodą zachłanną.
Dla symulowanego wyżarzania ustawiono temperaturę początkową, temperaturę końcową, współczynnik schładzania $a$ i prawdopodobieństwo akceptacji gorszego rozwiązania.
Dla TabuSearch ustawiono liczbę iteracji algorytmu, rozmiar listy tabu i kadencję ruchu.
\clearpage
\subsection{Symulowane wyżarzanie}
\begin{multicols}{3}

\begin{table}[H]
    \centering
    \begin{tabular}{|r|r|}
        \hline
        \textbf{Iteracja} & \textbf{Błąd ${[\%]}$}
        \csvreader[head to column names, filter expr={
            test{\ifnumless{\thecsvinputline}{25}}
        }]
        {./../../benchmark/tests/SA_48_1000_shift.csv}{}
        {\\\hline\csvcoli & \csvcoliii}\\
        \hline
    \end{tabular}
    \caption{Średnie czasy wykonania algorytmu wykorzystującego metodę symulowanego wyżarzania dla pliku ftv47. Pierwsze 25 wyników.}
\end{table}
\columnbreak
\begin{table}[H]
    \centering
    \begin{tabular}{|r|r|}
        \hline
        \textbf{Iteracja} & \textbf{Błąd ${[\%]}$}
        \csvreader[head to column names, filter expr={
            test{\ifnumless{\thecsvinputline}{25}}
        }]
        {./../../benchmark/tests/SA_171_1000_shift.csv}{}
        {\\\hline\csvcoli & \csvcoliii}\\
        \hline
    \end{tabular}
    \caption{Średnie czasy wykonania algorytmu wykorzystującego metodę symulowanego wyżarzania dla pliku ftv170. Pierwsze 25 wyników.}
\end{table}
\columnbreak
\begin{table}[H]
    \centering
    \begin{tabular}{|r|r|}
        \hline
        \textbf{Iteracja} & \textbf{Błąd ${[\%]}$}
        \csvreader[head to column names, filter expr={
            test{\ifnumless{\thecsvinputline}{25}}
        }]
        {./../../benchmark/tests/SA_403_1000_shift.csv}{}
        {\\\hline\csvcoli & \csvcoliii}\\
        \hline
    \end{tabular}
    \caption{Średnie czasy wykonania algorytmu wykorzystującego metodę symulowanego wyżarzania dla pliku rbg403. Pierwsze 25 wyników.}
\end{table}
\end{multicols}

\subsubsection{ftv47}

\begin{figure}[H]
    \centering
    \begin{tikzpicture}
        \begin{axis}[
        width=0.9\linewidth,
        height=7cm,
        grid=major,
        xlabel=Czas{[iteracje]},
        ylabel=Błąd{[\%]},
        legend style={anchor=north east, at={(1,1)}}
        ]
        \addplot table [x index=0, y index=2, col sep=comma] {./../../benchmark/tests/SA_48_1000_shift.csv};
        \addlegendentry{Insert}
        \addplot table [x index=0, y index=2, col sep=comma] {./../../benchmark/tests/SA_48_1000_swap.csv};
        \addlegendentry{Swap}
        \end{axis}
    \end{tikzpicture}
    \caption{Symulowane wyżarzanie. Wykres dla pliku ftv47. Dwie definicje sąsiedztwa.}
    \label{plot:1}
\end{figure}
Na wykresie z rysunku \ref{plot:1} przedstawiono tylko dwie definicje sąsiedztwa, ponieważ sąsiedztwo \textit{Reverse} nie poprawiło wyniku zachłannego poprzez cały czas działania algorytmu.

Najlepsza ścieżka została znaleziona w iteracji \textbf{1074093} z błędem \textbf{1.63\%}.

\subsubsection{ftv170}
\begin{figure}[H]
    \centering
    \begin{tikzpicture}
        \begin{axis}[
        width=0.9\linewidth,
        height=7cm,
        grid=major,
        xlabel=Czas{[iteracje]},
        ylabel=Błąd{[\%]},
        legend style={anchor=south west, at={(0,0)}}
        ]
        \addplot table [x index=0, y index=2, col sep=comma] {./../../benchmark/tests/SA_171_1000_shift.csv};
        \addlegendentry{Insert}
        \addplot table [x index=0, y index=2, col sep=comma] {./../../benchmark/tests/SA_171_1000_swap.csv};
        \addlegendentry{Swap}
        \end{axis}
    \end{tikzpicture}
    \caption{Symulowane wyżarzanie. Wykres dla pliku ftv170. Dwie definicje sąsiedztwa.}
    \label{plot:2}
\end{figure}
Na wykresie z rysunku \ref{plot:2} przedstawiono tylko dwie definicje sąsiedztwa, ponieważ sąsiedztwo \textit{Reverse} nie poprawiło wyniku zachłannego poprzez cały czas działania algorytmu.

Najlepsza ścieżka została znaleziona w iteracji \textbf{6041106} z błędem \textbf{17.06\%}.

\subsubsection{rbg403}
\begin{figure}[H]
    \centering
    \begin{tikzpicture}
        \begin{axis}[
        width=0.9\linewidth,
        height=7cm,
        grid=major,
        xlabel=Czas{[iteracje]},
        ylabel=Błąd{[\%]},
        legend style={anchor=south west, at={(0,0)}}
        ]
        \addplot table [x index=0, y index=2, col sep=comma] {./../../benchmark/tests/SA_403_1000_shift.csv};
        \addlegendentry{Insert}
        \addplot table [x index=0, y index=2, col sep=comma] {./../../benchmark/tests/SA_403_1000_swap.csv};
        \addlegendentry{Swap}
        \end{axis}
    \end{tikzpicture}
    \caption{Symulowane wyżarzanie. Wykres dla pliku rbg403. Dwie definicje sąsiedztwa.}
    \label{plot:3}
\end{figure}
Na wykresie z rysunku \ref{plot:3} przedstawiono tylko dwie definicje sąsiedztwa, ponieważ sąsiedztwo \textit{Reverse} nie poprawiło wyniku zachłannego poprzez cały czas działania algorytmu.

Najlepsza ścieżka została znaleziona w iteracji \textbf{30404254} z błędem \textbf{0.37\%}.

\clearpage
\subsection{TabuSearch}

\begin{multicols}{3}

    \begin{table}[H]
        \centering
        \begin{tabular}{|r|r|}
            \hline
            \textbf{Iteracja} & \textbf{Błąd ${[\%]}$}
            \csvreader[head to column names, filter expr={
                test{\ifnumless{\thecsvinputline}{25}}
            }]
            {./../../benchmark/tests/TS_48_1000000_shift.csv}{}
            {\\\hline\csvcoli & \csvcoliii}\\
            \hline
        \end{tabular}
        \caption{Średnie czasy wykonania algorytmu wykorzystującego metodę TabuSearch dla pliku ftv47. Pierwsze 25 wyników.}
    \end{table}
    \columnbreak
    \begin{table}[H]
        \centering
        \begin{tabular}{|r|r|}
            \hline
            \textbf{Iteracja} & \textbf{Błąd ${[\%]}$}
            \csvreader[head to column names, filter expr={
                test{\ifnumless{\thecsvinputline}{25}}
            }]
            {./../../benchmark/tests/TS_171_1000000_shift.csv}{}
            {\\\hline\csvcoli & \csvcoliii}\\
            \hline
        \end{tabular}
        \caption{Średnie czasy wykonania algorytmu wykorzystującego metodę TabuSearch dla pliku ftv170. Pierwsze 25 wyników.}
    \end{table}
    \columnbreak
    \begin{table}[H]
        \centering
        \begin{tabular}{|r|r|}
            \hline
            \textbf{Iteracja} & \textbf{Błąd ${[\%]}$}
            \csvreader[head to column names, filter expr={
                test{\ifnumless{\thecsvinputline}{25}}
            }]
            {./../../benchmark/tests/TS_403_1000000_shift.csv}{}
            {\\\hline\csvcoli & \csvcoliii}\\
            \hline
        \end{tabular}
        \caption{Średnie czasy wykonania algorytmu wykorzystującego metodę TabuSearch dla pliku rbg403. Pierwsze 25 wyników.}
    \end{table}
    \end{multicols}

\subsubsection{ftv47}
\begin{figure}[H]
    \centering
    \begin{tikzpicture}
        \begin{axis}[
        width=0.9\linewidth,
        height=7cm,
        grid=major,
        xlabel=Czas{[iteracje]},
        ylabel=Błąd{[\%]},
        legend style={anchor=north east, at={(1,1)}}
        ]
        \addplot table [x index=0, y index=2, col sep=comma] {./../../benchmark/tests/TS_403_1000000_reverse.csv};
        \addlegendentry{Reverse}
        \addplot table [x index=0, y index=2, col sep=comma] {./../../benchmark/tests/TS_48_1000000_shift.csv};
        \addlegendentry{Insert}
        \addplot table [x index=0, y index=2, col sep=comma] {./../../benchmark/tests/TS_48_1000000_swap.csv};
        \addlegendentry{Swap}
        \end{axis}
    \end{tikzpicture}
    \caption{TabuSearch. Wykres dla pliku ftv47. Trzy definicje sąsiedztwa.}
    \label{plot:5}
\end{figure}
Najlepsza ścieżka została znaleziona w iteracji \textbf{322419} z błędem \textbf{0.17\%}.

\subsubsection{ftv170}
\begin{figure}[H]
    \centering
    \begin{tikzpicture}
        \begin{axis}[
        width=0.9\linewidth,
        height=7cm,
        grid=major,
        xlabel=Czas{[iteracje]},
        ylabel=Błąd{[\%]},
        legend style={anchor=south east, at={(1,0)}}
        ]
        \addplot table [x index=0, y index=2, col sep=comma] {./../../benchmark/tests/TS_403_1000000_reverse.csv};
        \addlegendentry{Reverse}
        \addplot table [x index=0, y index=2, col sep=comma] {./../../benchmark/tests/TS_171_1000000_shift.csv};
        \addlegendentry{Insert}
        \addplot table [x index=0, y index=2, col sep=comma] {./../../benchmark/tests/TS_171_1000000_swap.csv};
        \addlegendentry{Swap}
        \end{axis}
    \end{tikzpicture}
    \caption{TabuSearch. Wykres dla pliku ftv170. Trzy definicje sąsiedztwa.}
    \label{plot:6}
\end{figure}
Najlepsza ścieżka została znaleziona w iteracji \textbf{58267} z błędem \textbf{11.36\%}.

\subsubsection{rbg403}
\begin{figure}[H]
    \centering
    \begin{tikzpicture}
        \begin{axis}[
        width=0.9\linewidth,
        height=7cm,
        grid=major,
        xlabel=Czas{[iteracje]},
        ylabel=Błąd{[\%]},
        legend style={anchor=north east, at={(1,1)}}
        ]
        \addplot table [x index=0, y index=2, col sep=comma] {./../../benchmark/tests/TS_403_1000000_reverse.csv};
        \addlegendentry{Reverse}
        \addplot table [x index=0, y index=2, col sep=comma] {./../../benchmark/tests/TS_403_1000000_shift.csv};
        \addlegendentry{Insert}
        \addplot table [x index=0, y index=2, col sep=comma] {./../../benchmark/tests/TS_403_1000000_swap.csv};
        \addlegendentry{Swap}
        \end{axis}
    \end{tikzpicture}
    \caption{TabuSearch. Wykres dla pliku rbg403. Trzy definicje sąsiedztwa.}
    \label{plot:7}
\end{figure}

\begin{figure}[H]
    \centering
    \begin{tikzpicture}
        \begin{axis}[
        width=0.9\linewidth,
        height=7cm,
        grid=major,
        xlabel=Czas{[iteracje]},
        ylabel=Błąd{[\%]},
        legend style={anchor=north east, at={(1,1)}},
         xmax=5000,
        ]
        \addplot table [x index=0, y index=2, col sep=comma] {./../../benchmark/tests/TS_403_1000000_reverse_copy.csv};
        \addlegendentry{Reverse}
        \addplot table [x index=0, y index=2, col sep=comma] {./../../benchmark/tests/TS_403_1000000_shift_copy.csv};
        \addlegendentry{Insert}
        \addplot table [x index=0, y index=2, col sep=comma] {./../../benchmark/tests/TS_403_1000000_swap_copy.csv};
        \addlegendentry{Swap}
        \end{axis}
    \end{tikzpicture}
    \caption{TabuSearch. Wykres dla pliku rbg403. Trzy definicje sąsiedztwa. Przybliżenie.}
    \label{plot:8}
\end{figure}

Najlepsza ścieżka została znaleziona w iteracji \textbf{922660} z błędem \textbf{0.57\%}.

\clearpage
\section{Podsumowanie}
Z dwóch analizowanych metod zdecydowanie lepiej radzi sobie metoda TabuSearch.

Spośród analizowanych definicji sąsiedztwa najgorszą z nich jest sąsiedztwo \textit{Reverse}. Nie pozwala ono na powstanie małej modyfikacji, która miałaby małe skutki, ponieważ dotyczy ona więcej niż dwóch elementów i ich sąsiadów.
Sąsiedztwo \textit{Reverse} źle radzi sobie w asymetrycznym problemie komiwojażera. Sąsiedztwa dla problemu symetrycznego byłby lepszej jakości, ponieważ zmiana kolejności na odwrotną nie powoduje tam zmiany długości drogi.

\subsection{Najlepsze znalezione ścieżki}

\subsubsection{ftv47}
Najlepsze rozwiązanie: 1776\\
Znalezione rozwiązanie: 1779\\
Błąd: 0.17\%\\
Ścieżka:\\
\tiny
21 -> 40 -> 47 -> 26 -> 42 -> 28 -> 3 -> 24 -> 4 -> 29 -> 30 -> 5 -> 31 -> 6 -> 8 -> 10 -> 11 -> 0 -> 25 -> 1 -> 9 -> 33 -> 27 -> 2 -> 41 -> 43 -> 22 -> 20 -> 38 -> 37 -> 18 -> 17 -> 12 -> 32 -> 7 -> 23 -> 34 -> 13 -> 46 -> 36 -> 35 -> 14 -> 15 -> 16 -> 45 -> 39 -> 19 -> 44 -> 21
\normalsize
\subsubsection{ftv170}
Najlepsze rozwiązanie: 2755\\
Znalezione rozwiązanie: 3068\\
Błąd: 11.36\%\\
Ścieżka:\\
\tiny
    134 -> 132 -> 110 -> 109 -> 108 -> 83 -> 84 -> 71 -> 60 -> 50 -> 51 -> 52 -> 53 -> 43 -> 55 -> 54 -> 58 -> 59 -> 68 -> 67 -> 63 -> 64 -> 56 -> 57 -> 62 -> 61 -> 66 -> 65 -> 88 -> 153 -> 166 -> 107 -> 106 -> 105 -> 98 -> 95 -> 94 -> 92 -> 91 -> 87 -> 70 -> 69 -> 167 -> 85 -> 86 -> 93 -> 154 -> 89 -> 90 -> 96 -> 97 -> 165 -> 163 -> 99 -> 120 -> 121 -> 122 -> 162 -> 123 -> 101 -> 100 -> 102 -> 103 -> 104 -> 114 -> 164 -> 127 -> 126 -> 125 -> 128 -> 130 -> 135 -> 138 -> 139 -> 140 -> 141 -> 6 -> 7 -> 8 -> 9 -> 10 -> 76 -> 74 -> 75 -> 11 -> 12 -> 13 -> 18 -> 19 -> 21 -> 20 -> 32 -> 158 -> 36 -> 157 -> 33 -> 31 -> 30 -> 28 -> 27 -> 26 -> 23 -> 24 -> 25 -> 150 -> 160 -> 151 -> 14 -> 15 -> 159 -> 16 -> 17 -> 29 -> 22 -> 37 -> 38 -> 39 -> 35 -> 156 -> 34 -> 155 -> 41 -> 42 -> 40 -> 45 -> 44 -> 46 -> 47 -> 48 -> 49 -> 170 -> 73 -> 77 -> 168 -> 72 -> 82 -> 78 -> 79 -> 80 -> 81 -> 0 -> 1 -> 2 -> 3 -> 4 -> 5 -> 169 -> 111 -> 112 -> 133 -> 131 -> 113 -> 115 -> 116 -> 117 -> 118 -> 119 -> 124 -> 137 -> 136 -> 129 -> 146 -> 145 -> 144 -> 143 -> 147 -> 148 -> 149 -> 161 -> 152 -> 142 -> 134
\normalsize
\subsubsection{rbg403}
Najlepsze rozwiązanie: 2465\\
Znalezione rozwiązanie: 2474\\
Błąd: 0.37\%\\
Ścieżka:\\
\tiny
277 -> 263 -> 397 -> 260 -> 90 -> 72 -> 266 -> 102 -> 182 -> 336 -> 321 -> 289 -> 286 -> 74 -> 309 -> 335 -> 176 -> 31 -> 6 -> 331 -> 323 -> 52 -> 40 -> 352 -> 169 -> 27 -> 268 -> 16 -> 47 -> 112 -> 244 -> 80 -> 179 -> 368 -> 238 -> 229 -> 14 -> 96 -> 94 -> 126 -> 10 -> 190 -> 138 -> 18 -> 139 -> 375 -> 371 -> 46 -> 82 -> 249 -> 240 -> 219 -> 109 -> 326 -> 210 -> 158 -> 1 -> 193 -> 108 -> 256 -> 247 -> 217 -> 384 -> 383 -> 9 -> 207 -> 45 -> 127 -> 129 -> 255 -> 385 -> 199 -> 77 -> 156 -> 390 -> 373 -> 135 -> 270 -> 157 -> 269 -> 151 -> 101 -> 213 -> 28 -> 283 -> 8 -> 354 -> 200 -> 337 -> 97 -> 153 -> 75 -> 246 -> 95 -> 271 -> 107 -> 61 -> 85 -> 32 -> 113 -> 281 -> 137 -> 36 -> 248 -> 180 -> 165 -> 202 -> 178 -> 100 -> 44 -> 92 -> 146 -> 320 -> 208 -> 128 -> 81 -> 22 -> 21 -> 188 -> 299 -> 298 -> 291 -> 379 -> 177 -> 38 -> 43 -> 34 -> 388 -> 348 -> 290 -> 181 -> 346 -> 66 -> 273 -> 370 -> 328 -> 357 -> 322 -> 317 -> 131 -> 355 -> 214 -> 192 -> 189 -> 65 -> 262 -> 261 -> 68 -> 160 -> 15 -> 93 -> 203 -> 237 -> 402 -> 287 -> 395 -> 253 -> 374 -> 363 -> 362 -> 304 -> 5 -> 142 -> 209 -> 41 -> 167 -> 239 -> 117 -> 124 -> 339 -> 105 -> 19 -> 122 -> 401 -> 400 -> 104 -> 349 -> 305 -> 215 -> 73 -> 171 -> 63 -> 377 -> 110 -> 33 -> 376 -> 316 -> 265 -> 275 -> 147 -> 99 -> 389 -> 187 -> 79 -> 353 -> 39 -> 211 -> 183 -> 17 -> 279 -> 194 -> 380 -> 315 -> 332 -> 296 -> 295 -> 152 -> 76 -> 399 -> 71 -> 251 -> 358 -> 350 -> 134 -> 288 -> 360 -> 69 -> 111 -> 369 -> 234 -> 228 -> 163 -> 25 -> 119 -> 319 -> 164 -> 3 -> 333 -> 267 -> 257 -> 345 -> 173 -> 344 -> 87 -> 125 -> 60 -> 98 -> 103 -> 276 -> 259 -> 141 -> 392 -> 351 -> 293 -> 313 -> 378 -> 297 -> 166 -> 300 -> 361 -> 51 -> 49 -> 37 -> 174 -> 250 -> 372 -> 356 -> 50 -> 56 -> 140 -> 334 -> 398 -> 396 -> 83 -> 42 -> 133 -> 64 -> 365 -> 245 -> 106 -> 359 -> 285 -> 282 -> 78 -> 226 -> 258 -> 231 -> 205 -> 204 -> 161 -> 148 -> 154 -> 314 -> 59 -> 170 -> 284 -> 307 -> 7 -> 172 -> 26 -> 220 -> 197 -> 233 -> 324 -> 144 -> 242 -> 70 -> 264 -> 274 -> 364 -> 303 -> 224 -> 159 -> 20 -> 86 -> 11 -> 91 -> 130 -> 252 -> 191 -> 162 -> 48 -> 2 -> 386 -> 391 -> 325 -> 155 -> 143 -> 149 -> 347 -> 341 -> 196 -> 0 -> 308 -> 394 -> 225 -> 301 -> 120 -> 116 -> 88 -> 201 -> 89 -> 186 -> 53 -> 168 -> 302 -> 118 -> 23 -> 393 -> 235 -> 121 -> 136 -> 236 -> 227 -> 212 -> 184 -> 4 -> 367 -> 310 -> 216 -> 342 -> 198 -> 222 -> 185 -> 312 -> 35 -> 230 -> 382 -> 381 -> 206 -> 30 -> 254 -> 223 -> 29 -> 150 -> 145 -> 340 -> 123 -> 175 -> 218 -> 221 -> 67 -> 338 -> 318 -> 366 -> 292 -> 329 -> 294 -> 327 -> 232 -> 387 -> 84 -> 272 -> 278 -> 57 -> 241 -> 330 -> 343 -> 195 -> 280 -> 12 -> 24 -> 13 -> 243 -> 311 -> 55 -> 132 -> 306 -> 115 -> 114 -> 62 -> 54 -> 58 -> 277
\normalsize
\end{document}