\documentclass[12pt]{article}
\usepackage{scrextend}
\usepackage[utf8]{inputenc}
\usepackage[polish]{babel}
\usepackage[T1]{fontenc}%polskie znaki
\usepackage[utf8]{inputenc}%polskie znaki
\usepackage{geometry}
\usepackage{float}
\usepackage{enumitem}
\usepackage{hyperref}
\usepackage{graphicx}
\usepackage{tabulary}
\usepackage{etoc}
\usepackage[normalem]{ulem} 
\renewcommand{\baselinestretch}{1.5}
\graphicspath{ {img/} }
\newgeometry{lmargin=2.0cm, rmargin=2.0cm, tmargin=2.0cm, bmargin=2.0cm}
\usepackage{tikz}
\usepackage[bf]{caption}
\usepackage{pgfplots}


\title{ 
    \vspace*{50mm}
    \textsc{
        \textbf{Projektowanie efektywnych algorytmów}\\
        \large Projekt 1
    }
} 
\author{
Damian Koper,  241292\\
}

\date{\today}

\begin{document}

\maketitle

\newpage
\setcounter{tocdepth}{2}
\localtableofcontents
\newpage

\newcommand{\threeplot}[3]{
    \begin{figure}[h]
        \centering        
        \begin{tikzpicture}
            \begin{#1}[
                width=0.9\linewidth,
                height=7cm,
                grid=major,
                xlabel=N,
                ylabel=T{[s]},
                xticklabel={%
                    \pgfmathtruncatemacro{\IntegerTick}{\tick}%
                    \pgfmathprintnumberto[verbatim,fixed,precision=3]{\tick}\tickAdjusted%
                    \pgfmathparse{\IntegerTick == \tickAdjusted ? 1: 0}%
                    \ifnum\pgfmathresult>0\relax$\IntegerTick$\else\fi%
                },
                legend style={#3}
            ]
            \addplot table [x=startElements, y=time, col sep=comma] {./../../benchmark/BruteForce.csv};
            \addlegendentry{Brute Force}
            \addplot table [x=startElements, y=time, col sep=comma] {./../../benchmark/BnB.csv};
            \addlegendentry{BnB}
            \addplot table [x=startElements, y=time, col sep=comma] {./../../benchmark/DP.csv};
            \addlegendentry{DP}
            \end{#1}
        \end{tikzpicture}
        \caption{#2}
    \end{figure}
}
\threeplot{axis}
{Porównanie czasów wykonania dla trzech metod.}
{anchor=north west, at={(0,1)}}
\threeplot{semilogyaxis}
{Porównanie czasów wykonania dla trzech metod. Skala logarytmiczna.}
{anchor=south east, at={(1,0)}}

\end{document}